\section{非线性迭代算法}

\subsection{简单混合迭代算法}

\subsection{Anderson混合迭代算法}
    Anderson 混合迭代算法主要用于求解不动点方程
	\begin{align*}
	u = G(u),
	\end{align*}
	其中$ u\in\mathbb{R}^n, G:\mathbb{R}^n\rightarrow \mathbb{R}^n. $算法框架如下
	\begin{algorithm}[!pbht]
		\caption{Anderson 混合迭代算法: Anderson(m)}
		\label{alg:anderson}
		\begin{algorithmic}[1]
			\REQUIRE {$ u_0, G, m $}
			\STATE $ u_1 = G(u_0) , F_0 = G(u_0) - u_0$;
			\FOR {$ k=1,2,\cdots $}
				\STATE $ m_k = \min(m, k) $;
				\STATE $ F_k = G(u_k) - u_k; $;
				\STATE Minimize $ \| \sum_{j=0}^{m_k}\alpha_j^kF_{k-m_k+j}\| $ subject to $ \sum_{j=0}^{m_k} \alpha^k_j = 1$;
				\STATE $u_{k+1}  = (1-\beta_k)\sum_{j=0}^{m_k}\alpha^k_ju_{k-m_k+j} + \beta_k\sum_{j=0}^{m_k}\alpha^k_jG(u_{k-m_k+j}) $;
			\ENDFOR			
		\end{algorithmic}
	\end{algorithm}
	\begin{remark}
		当$ m = 0 $时,算法 \ref{alg:anderson}就是简单的不动点迭代。
	\end{remark}
	\begin{remark}
		算法\ref{alg:anderson}中,混合参数$ \{\beta_k\} $一般启发式地给定。在后续讨论中,我们假设$ \beta_k\equiv 1 $。
	\end{remark}

	算法第$ k $步子问题中的范数可以取任意范数,特别地,如果取$ l^2 $范数,利用关系
	\begin{equation*}
	m_k = 1- \sum_{j=0}^{m_k-1} \alpha_j^k,
	\end{equation*}
	带约束的子问题可以转换为以下更容易求解的无约束问题
	\begin{align}
	\min_{\alpha = (\alpha_0^k,\cdots,\alpha_{m_k-1}^k)^T }\quad \left\| F(u_k) + \sum_{j=0}^{m_k-1} \alpha_j^k (F(u_{k-m_k+j} - F(u_k))) \right\|^2.
	\label{subpro}
	\end{align}
	如果取$ l^1 $或者$ l^\infty $范数,子问题等价于线性规划,从而有很多高效的算法求解。
	
	
	\paragraph{线性形式} 如果$ G $是线性的,即
	\begin{align*}
	G(u) = M(u) + b ,
	\end{align*}
	其中$ M $是线性算子并且$ \|M\|=c<1 $,此时残差
	\begin{align*}
	F(u) = G(u) -u = b - (I-M)u.
	\end{align*}
	下面定理表明,在线性情况下Anderson 迭代与广义极小残差法(GMRES)之间的等价关系。
	\begin{theorem}[Theorem 2 \cite{Homer}]
		对于任意$ k>0 $, $ r^{GMRES(m)} \neq 0$ 并且$$ \|r_{j-1}^{GMRES}\|_2> \|r_{j}^{GMRES}\|_2, \quad,j=1,2\cdots,k,$$ 那么
		\begin{align*}
		\sum_{j=0}^{m_k}\alpha^k_ju_k = u_k^{GMRES},\quad u_{k+1} = G(u_k^{GMRES}).
		\end{align*}
	\end{theorem}

	\begin{theorem}[Theorem 2.1 \cite{Alex}]
		如果$ \|M\| = c<1 $,那么 Anderson 迭代收敛到$ u^* = (I-M)^{-1}b $并且
		\begin{align*}
		\|u_k - u^*\| \leq \left(\dfrac{1+c}{1-c}\right)c^k \|u_0 - u^*\|,\quad k = 1,2,\cdots
		\end{align*}
	\end{theorem}

	\paragraph{非线性形式} 
	为了分析非线性情况下的收敛速度,我们对系数$ \alpha_j^k $做如下假设
	\begin{assumption}.
		\label{assm:alpha}
		\begin{itemize}
			\item $ \|\sum_{j=0}^{m_k}\alpha_j^kF(u_{k-m_k+j})\| \leq \|F(u_k)\|; $
			\item $ \sum_{j=0}^{m_k}\alpha_j^k = 1; $
			\item 	存在$M_\alpha  $使得$
			\sum_{j=0}^{m_k}|\alpha_j^k| \leq M_\alpha,\quad \forall k\geq 0.$
		\end{itemize}		
	\end{assumption}
	根据算法,前两项假设自然成立,第三项假设可以通过以下方法实现
	\begin{itemize}
		\item 当系数超过某个阈值,重启(Restart)迭代;
		\item 求解子问题时,加入关于系数的约束;
	\end{itemize}

	为了证明局部收敛性,我们还需要对$ G$的非线性性做如下假设,这是证明Newton法局部收敛速度的一般假设。
	\begin{assumption}.
		\label{assm:G}
		\begin{itemize}
			\item 存在$ u^* $使得$ F(u^*) =G(u^*) - u^* = 0; $
			\item $ G $在球$ \mathcal{B}(\hat{\rho}) = \{u|\|e\| = \|u-u^*\| \leq \hat{\rho}\} $上 Lipschitz连续可微;
			\item 存在常数$ c\in(0,1) $使得
			\begin{align*}
			\|G(u) - G(v)\|\leq c\|u - v\|,\quad \forall u,v\in\mathcal{B}(\hat{\rho}).
			\end{align*}
		\end{itemize}
	\end{assumption}

	令$G' $表示$G $的Jacobi矩阵,从上面的假设可知,$ \|G'\|\leq c<1 $,从而$ F'(u^*) $非奇异。
	
	\begin{theorem}[Theorem 2.3 \cite{Alex}]
		如果假设\ref{assm:G}立,取$ c<\hat{c}<1 $,那么当$ u $充分接近解$ u^* $时, Anderson迭代满足
		\begin{align*}
		\|F(u_k)\|\leq \hat{c}^k\|F(u_0)\|,\quad \|e_k\| \leq \dfrac{1+c}{1-c}\hat{c}^k\|e_0\|.
		\end{align*}
	\end{theorem}
	
	
	\paragraph{Anderson(1)} 如果$ m = 1 $并且将子问题的范数取成$ l^2 $范数,那么
	\begin{align*}
	u_{k+1} = (1-\alpha^k)G(u_k) + \alpha^k G(u_{k-1}),
	\end{align*}
	其中
	\begin{align*}
	\alpha^k = \dfrac{F(u_k)^T(F(u_k) - F(u_{k-1}))}{\|F(u_k) - F(u_{k-1})\|^2}.
	\end{align*}
	下面定理
	\begin{theorem}[Theorem 2.4 \cite{Alex}]
		如果假设\ref{assm:G}成立,$ u_0\in\mathcal{B}(\hat{\rho}) $并且$ c $足够小使得
		\begin{align*}
		\hat{c} = \dfrac{3c - c^2}{1-c}<1.
		\end{align*}
		那么 Anderson(1) 算法中残差满足
		\begin{align*}
		\|F(u_k)\| \leq \hat{c}\|F(u_{k-1})\| \leq \hat{c}^k\|F(u_0)\|.
		\end{align*}
	\end{theorem}

	上述定理表明系数$ \alpha^k $是有界,因为
	\begin{align*}
	|\alpha^k| \leq \dfrac{\|F(u_k)\|}{\|F(u_k) - F(u_{k-1})\|}\leq \dfrac{\hat{c}\|F(u_{k-1})\|}{(1-\hat{c})\|F(u_{k-1}\|)} \leq \dfrac{\hat{c}}{1-\hat{c}}
	\end{align*}
	
	\paragraph{Anderson 迭代与 拟Newton法}
	取子问题为$l^2  $范数,子问题(\ref{subpro})等价于
	\begin{align}
	\min_{\gamma = (\gamma_0,\cdots,\gamma_{m_k - 1})^T} \quad \|F_k - \calF_k\gamma\|_2,
	\label{subpro2}
	\end{align}
	其中$ \calF = (\Delta_F{k-m_k}, \cdots, \Delta F_{k-1}) ,\Delta F_i = F_{i+1} - F_i $。子问题(\ref{subpro})中的$ \alpha $与$ \gamma $满足
	\begin{align*}
	\alpha_0 = \gamma_0,\quad \alpha_i = \gamma_i - \gamma_{i-1}(1\leq i\leq m_k-1).
	\end{align*}
	令$ \gamma^{k} = (\gamma_0^k,\cdots,\gamma^k_{m_k-1} )^T$是子问题(\ref{subpro2})的解,如果$ \calF_k $是满秩的,那么
	\begin{align}
	\gamma^k = (\calF_k^T\calF_k)^{-1}\calF_k^TF_k.
	\label{gamma}
	\end{align}
	于是Anderson迭代为
	\begin{align*}
	u_{k+1}& = G(u_k) - \sum_{i=0}^{m_k-1}\gamma_i^k[G(u_{k-m_k+i+1}) - G(x_{k-m_k + i})]\\
	& = u_k + F_k -(\calX_k + \calF_k)\gamma^k ,
	\end{align*}
	其中$\calX_k= (\Delta u_{k-m_k},\cdots,\Delta u_{k-1})  ,\Delta u = u_{i+1} - u_i.$ 将(\ref{gamma})代入,得到 Anderson迭代的拟Newton矩阵形式
	\begin{align}
	u_{k+1} =  u_k - H_kF_k
	\end{align}
	其中$ H_k = -I +  (\calX_k + \calF_k)(\calF_k^T\calF_k)^{-1}\calF_k^T$是拟Newton矩阵

\subsection{交替方向迭代算法}
{\color{red}\begin{center}
    王鑫
\end{center}}

\subsection{拟牛顿半隐格式}
