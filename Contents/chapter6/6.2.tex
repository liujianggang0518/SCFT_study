\section{时间离散方法}

\subsection{Euler方法}
{\color{red}\begin{center}
     高晨阳
\end{center}}
Euler方法是最简单的数值方法,我们以常微分方程初值问题
\begin{equation}\label{eq:ODE}
		\begin{cases}
				y'=f(x,y), \quad\quad a\le x\le b , \quad |y| < \infty ,\\
				y(a)=y_0
		\end{cases}
\end{equation}
为例进行说明。其中$f$为$x,y$的已知函数,$y_0$为给定的初值。

在$[a,b]$中插入分点
$$ a = x_0 < x_1 <x_2 < ... < x_N = b ,$$
记$h_m = x_{m+1}- x_{m}$, $h_m$称为步长,如无特殊说明,下面总是假定$h_m = h$
为定步长。

我们的目的是寻求问(\ref{eq:ODE})在这一系列节点$x_0, x_1, ... ,x_N$上的近似解$y_1, y_2,... , y_N$. 为此,将(\ref{eq:ODE})的微分方程写成等价的积分方程形式:
\begin{equation}\label{eq:ODE_quard}
		y(x+h)=y(x)+\int_x^{x+h} f(\tau, y(\tau)) \, d\tau
\end{equation}
在式(\ref{eq:ODE_quard})中,令$x=x_m$,并用左矩形公式计算右端积分,得到
\begin{equation}\label{eq:ODE_dis}
		y(x_m+h)= y(x_m) +h f(x_m, y(x_m)) + R_m
\end{equation}
其中$R_m$称为余项。
\begin{equation}\label{eq:Rm}
		R_m=\int_{x_m}^{x_{m+1}}f(x, y(x))dx- h f(x_m,y(x_m))
\end{equation}
在式(\ref{eq:ODE_dis})中截去余项$R_m$, 便得近似计算公式
\begin{equation}\label{eq:ODE_solution}
		y(x_{m+1}) \approx y(x_m) + h f(x_m,y(x_m))
\end{equation}
由于除$m=0$外,$y(x_m)$是未知的,设$y_m$为$y(x_m)$的近似值,以$y_m+h f( x_m, y_m)$作为$ y(x_{m+1})$的近似值,记为$y_{m+1}$, 则得出求各个节点处解的近似值的递推公式。
\begin{equation}
		y_{m+1} = y_m + h f(x_m , y_m), \quad \quad m = 0, 1, ... ,N-1.
\end{equation}
这便是Euler方法。

同样地,在(\ref{eq:ODE_quard})中令$x = x_m$, 并分别用右矩形公式和梯形公式计算右端积分,且作同样的处理后,得到递推公式:
\begin{equation}\label{eq:ODE_implicit}
		\begin{aligned}
				y_{m+1}=y_m + h f(x_{m+1} , y_{m+1})\\
		        R_m = \int_xm^x_{m+1} f(x,y(x)) dx - h f(x_{m+1}, y_{m+1})
        \end{aligned}
\end{equation}
\begin{equation}\label{eq:ODE_modifty}
		\begin{aligned}
				y_{m+1}= y_m + h \frac{f(x_m,y_m) + f(x_{m+1},y_{m+1})}{2},\\
				R_m = \int_{x_m}^{x_{m+1}} f(x,y(x))dx- \frac{h}{2}[f(x_m,y(x_m))+ f(x_{m+1},y(x_{m+1}))]
		\end{aligned}
\end{equation}
我们称(\ref{eq:ODE_implicit})为隐式Euler方法, 而(\ref{eq:ODE_modifty})常称为改进的Eulur方法。
\subsubsection{Eulur方法的局部截断误差}
进一步假设$f(x,y)$关于$x$满足Lipschitz条件,K为Lipschitz常数,则由(\ref{eq:Rm})得
\begin{equation*}
		\begin{aligned}
				|R_m|= & | \int_{x_m}^{x_{m+1}} [f(x,y(x)) - f(x_m,y(x_m))] dx|\\
				&\leq \int_{x_m}^{x_{m+1}}|f(x,y(x)) - f(x_m, y(x))| dx\\
				&+\int_{x_m}^{x_{m+1}}|f(x_m,y(x)) - f(x_m, y(x_m))| dx\\
				&\leq K \int_{x_m}^{x_{m+1}}|x-x_m|dx + L\int_{x_m}^{x_{m+1}}|y(x)-y(x_m)|dx\\
				&\leq \frac{Kh^2}{2}+ L\int_{x_m}^{x_{m+1}}|y'(x_m+\theta (x-x_m))|(x-x_m)dx\\
				&\leq \frac{(K+LM)h^2}{2}
		\end{aligned}
\end{equation*}

其中$0 <\theta < 1$,$ M = \max\limits_{x \in [a,b]}|y'(x)| = \max\limits_{x\in[a,b]}|f(x,y(x))|$,记$R= (K+LM)h^2/2$, 则有
$$ |R_m| \leq R $$
由此可知,Euler方法的局部截断误差是二阶的。

\subsection{Runge-Kutta方法}
{\color{red}\begin{center}
     高晨阳
\end{center}}

\subsection{线性多步法}
{\color{red}\begin{center}
     司伟
\end{center}}
\subsubsection{固定步长}
\subsubsection{变步长}



\subsection{算子分裂法}
{\color{red}\begin{center}
    王鑫
\end{center}}


\subsection{谱延迟矫正方法}
{\color{red}\begin{center}
    王鑫
\end{center}}



