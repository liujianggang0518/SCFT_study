\section{时间离散方法}

\subsection{Euler方法}
{\color{red}\begin{center}
     高晨阳
\end{center}}
Euler方法是最简单的数值方法,我们以常微分方程初值问题
\begin{equation}\label{eq:ODE}
		\begin{cases}
				y'=f(x,y), \quad\quad a\le x\le b , \quad |y| < \infty ,\\
				y(a)=y_0
		\end{cases}
\end{equation}
为例进行说明。其中$f$为$x,y$的已知函数,$y_0$为给定的初值。

在$[a,b]$中插入分点
$$ a = x_0 < x_1 <x_2 < ... < x_N = b ,$$
记$h_m = x_{m+1}- x_{m}$, $h_m$称为步长,如无特殊说明,下面总是假定$h_m = h$
为定步长。

我们的目的是寻求问(\ref{eq:ODE})在这一系列节点$x_0, x_1, ... ,x_N$上的近似解$y_1, y_2,... , y_N$. 为此,将(\ref{eq:ODE})的微分方程写成等价的积分方程形式:
\begin{equation}\label{eq:ODE_quard}
		y(x+h)=y(x)+\int_x^{x+h} f(\tau, y(\tau)) \, d\tau
\end{equation}
在式(\ref{eq:ODE_quard})中,令$x=x_m$,并用左矩形公式计算右端积分,得到
\begin{equation}\label{eq:ODE_dis}
		y(x_m+h)= y(x_m) +h f(x_m, y(x_m)) + R_m
\end{equation}
其中$R_m$称为余项。
\begin{equation}\label{eq:Rm}
		R_m=\int_{x_m}^{x_{m+1}}f(x, y(x))dx- h f(x_m,y(x_m))
\end{equation}
在式(\ref{eq:ODE_dis})中截去余项$R_m$, 便得近似计算公式
\begin{equation}\label{eq:ODE_solution}
		y(x_{m+1}) \approx y(x_m) + h f(x_m,y(x_m))
\end{equation}
由于除$m=0$外,$y(x_m)$是未知的,设$y_m$为$y(x_m)$的近似值,以$y_m+h f( x_m, y_m)$作为$ y(x_{m+1})$的近似值,记为$y_{m+1}$, 则得出求各个节点处解的近似值的递推公式。
\begin{equation}
		y_{m+1} = y_m + h f(x_m , y_m), \quad \quad m = 0, 1, ... ,N-1.
\end{equation}
这便是Euler方法。

同样地,在(\ref{eq:ODE_quard})中令$x = x_m$, 并分别用右矩形公式和梯形公式计算右端积分,且作同样的处理后,得到递推公式:
\begin{equation}\label{eq:ODE_implicit}
		\begin{aligned}
				y_{m+1}=y_m + h f(x_{m+1} , y_{m+1})\\
		        R_m = \int_xm^x_{m+1} f(x,y(x)) dx - h f(x_{m+1}, y_{m+1})
        \end{aligned}
\end{equation}
\begin{equation}\label{eq:ODE_modifty}
		\begin{aligned}
				y_{m+1}= y_m + h \frac{f(x_m,y_m) + f(x_{m+1},y_{m+1})}{2},\\
				R_m = \int_{x_m}^{x_{m+1}} f(x,y(x))dx- \frac{h}{2}[f(x_m,y(x_m))+ f(x_{m+1},y(x_{m+1}))]
		\end{aligned}
\end{equation}
我们称(\ref{eq:ODE_implicit})为隐式Euler方法, 而(\ref{eq:ODE_modifty})常称为改进的Eulur方法。
\subsubsection{Eulur方法的局部截断误差}
进一步假设$f(x,y)$关于$x$满足Lipschitz条件,K为Lipschitz常数,则由(\ref{eq:Rm})得
\begin{equation*}
		\begin{aligned}
				|R_m|= & | \int_{x_m}^{x_{m+1}} [f(x,y(x)) - f(x_m,y(x_m))] dx|\\
				&\leq \int_{x_m}^{x_{m+1}}|f(x,y(x)) - f(x_m, y(x))| dx\\
				&+\int_{x_m}^{x_{m+1}}|f(x_m,y(x)) - f(x_m, y(x_m))| dx\\
				&\leq K \int_{x_m}^{x_{m+1}}|x-x_m|dx + L\int_{x_m}^{x_{m+1}}|y(x)-y(x_m)|dx\\
				&\leq \frac{Kh^2}{2}+ L\int_{x_m}^{x_{m+1}}|y'(x_m+\theta (x-x_m))|(x-x_m)dx\\
				&\leq \frac{(K+LM)h^2}{2}
		\end{aligned}
\end{equation*}

其中$0 <\theta < 1$,$ M = \max\limits_{x \in [a,b]}|y'(x)| = \max\limits_{x\in[a,b]}|f(x,y(x))|$,记$R= (K+LM)h^2/2$, 则有
$$ |R_m| \leq R $$
由此可知,Euler方法的局部截断误差是二阶的。

\subsection{Runge-Kutta方法}
{\color{red}\begin{center}
     高晨阳
\end{center}}
\subsubsection{(1)显式方法}

考察差商$\frac{y(x_{m+1})-y(x_m)}{h}$,根据微分中值定理,存在点$\xi$, $x_m < \xi <x_{m+1}$, 使得
$$y(x_m+1)- y(x_m)=h y'(\xi)$$
从而利用所给方程$y' = f(x,y)$, 得
\begin{equation}\label{eq:quot}
        y(x_{m+1}) = y(x_m) + h f(\xi,y(\xi))
\end{equation}
记$ k = f(\xi,y(\xi))$, 称$k$为区间$[x_m,x_{m+1}]$上曲线$y = y(x)$的平均斜率。只要对平均斜率提供一种算法,由式(\ref{eq:quot})便相应的导出微分方程的一种数值计算格式。例如,取点$x_m$的斜率值$k_1= f(x_m ,y_m )$作为平均斜率$k$, 则得到前面讨论的$Euler$方法,其精度当然很低。

又如,改进的$Euler$方法(\ref{eq:ODE_modifty})可以改写成下面的平均形式:
\begin{equation}
		\begin{cases}
				y_{m+1} = y_m +h(k1+k2)/2\\
				k_1 = f(x_m,y_m)\\
				k_2 = f(x_{m+1} , y_m+hk_1)
		\end{cases}
\end{equation}
因此可以理解为:它用$x_m$与$x_{m+1}$两个点的斜率值$k_1$和$k_2$,取算术平均作为平均斜率$k$,而$x_{m+1}$处的斜率值$k_2$则利用已知信息通过Euler公式来预报。这各处理过程启发我们,如果设法在$[x_m,x_{m+1}]$内多预报几个点的斜率值,然后将它们加权平均作为平均斜率$k$,则有可能构造出具有更高精度的计算格式,这就是Runge-Kutta方法的基本思想。根据这种思想,令
\begin{equation}\label{eq:formation}
		y_{m+1}= y_m + h \sum_{i=1}^s b_i k_i
\end{equation}
其中$b_i$为待定权因子,s为所使用的斜率值的个数,$k_i$满足下列方程
\begin{equation}\label{eq:para}
		k_i = f(x_m+c_i h , y_m +h \sum_{j=1}^{i-1} a_{ij}k_j) , \quad\quad i = 1, ... s
\end{equation}
且
$$ c_1 = 0 , c_i = \sum_{j=1}^{i-1}a_{ij}, \quad\quad i = 2 , ... ,s$$
当$s =1$时,(\ref{eq:formation})和(\ref{eq:para})即前面讨论的Euler方法。(\ref{eq:formation})和(\ref{eq:para})中出现的参数$b_1,b_2,...,b_s,c_1,c_2,...c_s$及$a_{ij}$为提高方法的精度创造了条件。Runge-Kutta方法中的系数可以这样来决定。设$y(x)$满足微分方程,将式(\ref{eq:para})的右端在$x_m$点展开成关于h的Taylor级数,然后代入(\ref{eq:formation})中,同时将$y(x_m+h)$在$x_m$点展开成Taylor级数,使左右两端h的次数不超过q的项的系数对应相等,就得到确定系数$b_i, c_i, a_{ij}$的方程组,求解它的解也就得到q阶的Runge-Kutta方法的算式。

现以$s=2$的情形为例,来构造Runge-Kutta方法。

设为微分方程的真解$y(x_m)$具有充分高阶导数,将$y(x_m+h)$及$k_1,k_2$在$x_m$点展开成$Taylor$公式:
\begin{equation*}
		\begin{aligned}
				y(x_m+h) =& y(x_m) +hy'(x_m) +\frac{h^2}{2!}y''(x_m)+\frac{h^3}{3!}y'''(x_m)+O(h^4)\\
				= & y(x_m)+hf+\frac{h^2}{2!}(f_x+f_y \cdot f)+\frac{h^3}{3!}(f_{xx}+2f \cdot f_{xy}+f^2 \cdot f_{yy}+f_x f_y +f^2_y \cdot f)+O(h^4)\\
				k_1=&f(x_m,y_m)\\
				k_2= &f(x_m+c_2h,y_m+ha_{21}k_1)\\
				= & f(x_m,y_m)+h(c_2 f_x+a_{21}f \cdot f_y) + O(h^2)
		\end{aligned}
\end{equation*}
其中$f,f_x,f_y,f_{xx}$分别表示f(x,y)及其偏导数在点$(x_m,y_m)$的值。将$k_1,k_2$的展开式代入(\ref{eq:formation})的左边,并与$y(x_m+h)$的展开式比较的展开式比较,令$h,h^2$项系数相等,便得到
\begin{equation*}
		\begin{cases}
				b_1+b_2=1\\
				b_2c_2=1/2\\
				b_2a_{21}=1/2
		\end{cases}
\end{equation*}
取$c_2$作为自由参数便可以确定出$b_1,b_2,a_{21}$.特别地,若分别取$c_2$为$1/2,2/3$和$1$,就得到相应的$(b_1,b_2,a_{21})$分别为$(0,1,1/2),(1/4,3/4,2/3),(1/2,1/2,1)$, 相应的Runge-Kutta方法为
\begin{equation*}
		\begin{cases}
				y_{m+1} =y_m +h f\bigg(x_m+\frac{1}{2}h, y_m+\frac{1}{2}hf_m\bigg)\\
				y_{m+1} =y_m +\frac{1}{4}h\bigg[ f(x_m,y_m)+3f(x_m+\frac{2}{3}h, y_m+\frac{2}{3}hf_m)\bigg]\\
				y_{m+1} =y_m +\frac{1}{2}h [f(x_m,y_m)+f(x_m+h,y_m+hf_m)]
		\end{cases}
\end{equation*}
其中$f_m=f(x_m,y_m)$.这是三个典型的二阶Runge-Kutta方法,它们分别被称为中点公式、Heun公式和改进的Euler公式。

对于$s=3,4$的情形,我们可以完全仿上述的方法推导出三阶和四阶Runge-Kutta公式,下面给出几个著名的公式列举如下:

三级三阶显式Heun公式:
\begin{equation*}\label{eq:Heun_3}
		\begin{cases}
				y_{m+1}=y_m + h(k_1+3k_3)/4\\
				k_1=f(x_m,y_m)\\
				k_2=f(x_m+h/3,y_m+hk_1/3)\\
				k_3=f(x_m+2h/3,y_m+2hk_2/3)
		\end{cases}
\end{equation*}

三级三阶显式Kutta公式:
\begin{equation*}\label{eq:Kutta_3}
		\begin{cases}
				y_{m+1}=y_m + h(k_1+4k_2+k_3)/6\\
				k_1=f(x_m,y_m)\\
				k_2=f(x_m+h/2,y_m+hk_1/2)\\
				k_3=f(x_m+h,y_m-hk_1+2hk_2)
		\end{cases}
\end{equation*}

四级四阶古典显式Runge-Kutta公式:
\begin{equation*}\label{eq:Runge-Kutta_3}
		\begin{cases}
				y_{m+1}=y_m + h(k_1+2k_2+2k_3+k_4)/6\\
				k_1=f(x_m,y_m)\\
				k_2=f(x_m+h/2,y_m+hk_1/2)\\
				k_3=f(x_m+h/2,y_m+hk_2/2)\\
                                k_4=f(x_m+h,y_m+hk_3)
		\end{cases}
\end{equation*}

四级四阶古典显式Kutta公式:
\begin{equation*}\label{eq:Kutta_4}
		\begin{cases}
				y_{m+1}=y_m + h(k_1+3k_2+3k_3+k_4)/8\\
				k_1=f(x_m,y_m)\\
				k_2=f(x_m+h/3,y_m+hk_1/3)\\
				k_3=f(x_m+2h/3,y_m-hk_1/3+hk_2)\\
                                k_4=f(x_m+h,y_m+hk_1-hk_2+hk_3)
		\end{cases}
\end{equation*}

四级四阶古典显式Gill公式:
\begin{equation*}\label{eq:Gill_4}
		\begin{cases}
				y_{m+1}=y_m + h(k_1+(2-\sqrt{2})k_2+(2+\sqrt{2})k_3+k_4)/6\\
				k_1=f(x_m,y_m)\\
				k_2=f(x_m+h/2,y_m+hk_1/2)\\
				k_3=f(x_m+h/2,y_m+\frac{\sqrt{2}-1}{2}hk_1+(1-\frac{\sqrt{2}}{2})hk_2)\\
                                k_4=f(x_m+h,y_m-\frac{\sqrt{2}}{2}hk_2+(1+\frac{\sqrt{2}}{2})hk_3)
		\end{cases}
\end{equation*}

\subsubsection{(2)隐式方法}

上面的公式有一个共同的特点,就是在计算$k_{i+1}$时,只用到$k_1,k_2,...,k_i$.所以,上述公式都被称为显式Runge-Kutta方法。如果把(\ref{eq:para})改写成
\begin{equation}\label{implit}
		k_i = f\bigg( x_m+c_i h , y_m+ h\sum_{j=1}^sa_{ij}k_j\bigg), \quad i=1,2,..s
\end{equation}
且
$$\sum_{j=1}^s a_{ij} = c_i$$
则每个$k_i$中含有$k_1,...,k_s$,这时称Runge-Kutta方法(\ref{eq:formation})$\sim$ (\ref{implit})为隐式的,它的待定系数可以表示如下:
\begin{table}[H]
		\centering
		\begin{tabular}{c|cccc}
				$c_1$ & $a_{11}$ & $a_{12}$ & ... & $a_{1s}$\\
		        $c_2$ & $a_{21}$ & $a_{22}$ & ... & $a_{2s}$\\
		        \vdots & \vdots & \vdots &   & \vdots \\
		       $c_s$ & $a_{s1}$ & $a_{s2}$ & ... & $a_{ss}$\\ \hline
		            &  $b_1$   & $b_2$   &  ...& $b_s$
		\end{tabular}
\end{table}
记A$=(a_{ij})_{s \times s}, c=(c_1,c_2,...c_s)^T, b=(b_1,b_2,...,b_s)^T$,则Runge-Kutta方法(\ref{eq:formation})$\sim$ (\ref{implit})可以简记为
\begin{table}[htbp!]\label{coff}
		\centering
		\begin{tabular}{c|c}
				c & A\\ \hline
				  & $b^T$ 
		\end{tabular}
\end{table}
以后均用(\ref{coff})表示Runge-Kutta方法,而且前面讨论过的显式Runge-Kutta方法
对应A为主对角线上元素为零的下三角阵,因而它们也可以表示为(\ref{coff})的形式。如果A是一个主对角元素为非零的下三角矩阵,相应的Runge-Kutta方法称为半隐式的;如果A为一般的s级方阵,相应的公式是全隐式的。隐式方法也可以像显式方法的推导方法一样,用Taylor展开的方法来构造。下面是几个常用的隐式方法。

中点公式($m=1,p=2$):
\begin{table}[htbp!]\label{coff1}
		\centering
		\begin{tabular}{c|c}
				$\frac{1}{2}$& $\frac{1}{2}$\\ \hline
				  & $1$ 
		\end{tabular}\caption{隐式中点公式}
\end{table}
\newpage
隐式Runge-Kutta公式($m=2,p=4$):
\begin{table}[htbp!]\label{coff2}
		\centering
		\begin{tabular}{c|cc}
				$\frac{3-\sqrt{3}}{6}$& $\frac{1}{4}$ & $\frac{3-2\sqrt{3}}{12}$\\ 
				$\frac{3+\sqrt{3}}{6}$ & $\frac{3+2\sqrt{3}}{12}$  &$ \frac{1}{4}$ \\\hline
                                                        &$ \frac{1}{2}$ &$ \frac{1}{2}$
		\end{tabular}\caption{隐式Runge-Kutta公式}
\end{table}

Kuntzmann和Butcher公式($m=2,p=4$):
\begin{table}[htbp!]\label{coff3}
		\centering
		\begin{tabular}{c|ccc}
				$\frac{5-\sqrt{15}}{10}$&$ \frac{5}{36}$ &$ \frac{10-3\sqrt{15}}{45}$ &$ \frac{25-6\sqrt{15}}{180}$\\ 
				$\frac{1}{2}$& $\frac{10+3\sqrt{15}}{72}$ &$\frac{2}{9}$ & $\frac{10-23\sqrt{15}}{72}$ \\
                                $\frac{5+\sqrt{15}}{10}$& $\frac{25+6\sqrt{15}}{180}$ & $\frac{10+3\sqrt{15}}{45}$ & $\frac{5}{36}$\\\hline
                                                        & $\frac{5}{18}$ & $\frac{4}{9}$ &$\frac{5}{18}$
		\end{tabular}\caption{Kuntzmann和Butcher公式}
\end{table}
\subsection{线性多步法}
{\color{red}\begin{center}
     司伟
\end{center}}
\subsubsection{固定步长}
\subsubsection{变步长}



\subsection{算子分裂法}
{\color{red}\begin{center}
    王鑫
\end{center}}


\subsection{谱延迟矫正方法}
{\color{red}\begin{center}
    王鑫
\end{center}}



