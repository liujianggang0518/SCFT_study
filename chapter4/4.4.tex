\subsection{模型E:两嵌段共聚物熔体}

\begin{figure}[H]
	\centering   
	\includegraphics[width=12cm]{./figures/2.png}
	\caption{模型E中所考虑的嵌段共聚物熔体A型(暗)和B型(亮)均为连续高斯链。熔体是局部不可压缩的。}
	\label{模型E}
\end{figure}
模型C可以直接适用于处理$AB$两嵌段共聚物的不可压缩熔体(见图\ref{模型E})。模型$E$是文献中采用的嵌段共聚物的“标准”模型。二元均聚物共混体和两嵌段共聚物熔体的基本区别是后者是单组分体系。因此,只有一个单一配分函数$Q[w_{A},w_{B}]$描述双块熔体。此函数在$(3.98)$中定义,可用$(3.104)$表示为前向传播子$q$或“互补”传播子$q_{c}$。 对于具有$N$个统计段的$n$个双块链,通过将$(3.97)$推广到一个多链系统,得到微观段密度算子:\\
\begin{equation}
\hat{ \rho }_{A}(\br)=\sum_{j=1}^{n} \int _{0}^{fN}  \delta (\br-\br_{j}(s))~d s
\end{equation}
\begin{equation}
\hat{ \rho }_{B}(\br)=\sum_{j=1}^{n} \int _{fN}^{N} \delta (\br-\br_{j}(s))~d s 
\end{equation}
其中$f$表示属于$A$块统计段的份数。如果每个物种的片段定义为在熔体中占据的公共体积$\upsilon_0$,那么$f$也可以解释为$A$型段所占的平均体积份数。\\

采用与$C$模型相同的相互作用模型,即$\hat{\rho}_{+}=\hat{\rho}_{A}+\hat{\rho}_{B}$的不可压缩约束和与$\chi_{AB} \hat{\rho}_{A} \hat{\rho}_{B}$成正比的局部$A-B$相互作用,使共混物中的$(4.97)$易于追溯到共聚物熔体中。模型$E$场理论描述了$AB$两嵌段共聚物的不可压缩熔体:\\
\begin{equation}
Z_{C}(n,V,T)=Z_{0} \int \mathcal{D} w_{+} \int  \exp (-H[w_{+},w_{-}])~\mathcal{D} w_{-}
\end{equation}
其中$Z_0$表示$n$个非相互作用的连续高斯链的理想气体分配函数,并给出了有效哈密顿量\\
\begin{equation}
H[ w_{+}, w_{-}]= \rho_{0} \int  [(1/\chi_{AB})w_{-}^{2}-i w_{+}]-n \ln Q[w_{A},w_{B}]~d \br
\end{equation}
复场$w_{K}$的解释与$(4.98)$相同,单链分配函数$Q[w_{A},w_{B}]$按$(3.104)$计算,分别求解两个复扩散方程$(3.99)$或$(3.101)$。两嵌段共聚物的功能$Q$与进入模型$C$的功能$Q_{K}$ 具有明显不同的非局部性质,这种差异是由于共聚物的$A$、$B$嵌段连通性造成的。\\

大正则模型E场理论是通过简单地改变有效哈密顿量来定义的\\
\begin{equation}
H_{G}[ w_{+}, w_{-}]= \rho_{0} \int  [(1/\chi_{AB}) w_{-}^{2}-i w_{+}]- z  V Q[w_{A},w_{B}]~d\br
\end{equation}
其中$z$是二嵌段共聚物的活性。\\

模型$E$的单体密度算子在模型$C$中被定义,但由于单链分配函数的不同,所以计算方法不同。它是从正则系综中的$(3.105)$、$(3.106)$和$(4.102)$中得到的,\\
\begin{equation}
\begin{aligned}
\tilde{\rho}_{K}(r;[w_{A},w_{B}]) &=-n \frac{\delta \ln Q[w_{A},w_{B}]}{\delta w_{K}(r)} \\&= \frac{n}{V Q[w_{A},w_{B}]} \int_{D_{K}} q_{c}(r,N-s;[w_{A},w_{B}])q(r,s;[w_{A},w_{B}])~ds
\end{aligned}
\end{equation}
其中$\int_{D_{K}} ds$表示在$A$块上积分时,$s\in[0,fN]$,$K=A$,在$B$块上积分时,$s\in(fN,N]$,$K=B$。同样,在大正则系综中,\\
\begin{equation}
\begin{aligned}
\tilde{\rho}_{K,G}(r;[w_{A},w_{B}])&=-z V \frac{\delta Q[w_{A},w_{B}]}{\delta w_{K}(\br)} \\&= z \int_{D_{K}}  q_{c}(\br,N-s;[w_{A},w_{B}])q(\br,s;[w_{A},w_{B}])~d s
\end{aligned}
\end{equation}

虽然模型$E$是描述柔性$AB$两嵌段共聚物熔体的标准,但通过修改底层链或相互作用模型,可以生成许多相关的模型。对于链模型的修正,如珠弹簧二嵌段共聚物链,$Q$和密度算子的表达式可以用第3章中的公式很容易地适应。对于相互作用模型的改变,不涉及$H$和$H_{G}$中的$Q$项都可以修改。例如,将$D$模型$(4.106)$中与$\zeta_{-1}$成正比的项加入到$H$或$H_{G}$的模型$E$表达式中,就得到了可压缩$AB$二嵌段共聚物熔体的模型。

