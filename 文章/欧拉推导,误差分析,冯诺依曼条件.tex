\documentclass{article}
\usepackage{latexsym,amsmath,amssymb,tabu,CJKutf8,graphicx,bm,blkarray,delarray,float}
\textheight 9in \textwidth 6.5in \oddsidemargin 0pt \topmargin -8pt
\pagestyle{plain}


\newcommand{\phit}{\phi_{t+\Delta t}}
\begin{document}
\begin{CJK}{UTF8}{gkai}
设初值问题
\begin{equation*}
\begin{cases}
u^{'}=f(t,u),\\
u(t_{0})=u_{0}
\end{cases}
\end{equation*}
其中$u_{0}$是给定的初值,是一阶常微分方程的初值问题.

将区间$[0,T]$作N等分.小区间的长度$h=\frac{T}{N}$称为步长.点列$t_{n}=hn(n=0,1,\cdots,N)$称为节点.

已知初值$u(t_{0})=u_{0}$即有
\begin{equation}
u^{'}(t_{0})=f(t_{0},u(t_{0}))=f(t_{0},u_{0})
\end{equation}
$u(t_{1})$在$u(t_{0})$处泰勒展开:
\begin{align}
u(t_{1})&=u(t_{0}+h)\\&=u(t_{0})+hu^{'}(t_{0})+\frac{h^2}{2}u^{''}(\xi)\\&=u_{0}+hf(t_{0},u_{0})+R_{0}~~~\xi\in (t_{0},t_{1})
\end{align}
$R_{0}$称为欧拉的局部截断误差.略去二阶小量$R_{0}$得:
\begin{equation}
u_{1}=u_{0}+hf(t_{0},u_{0})
\end{equation}
$u_{1}$是$u(t_{1})$的近似值,一般递推公式为:
\begin{equation}
u_{n+1}=u_{n}+hf(t_{n},u_{n})
\label{1}
\end{equation}
这是向前欧拉法.
~\\
~\\
~\\
\par 
\textbf{误差分析:}
一般的递推式为:
\begin{equation}
u(t_{n+1})=u(t_{n})+hf(t_{n},u(t_{n}))+R_{n}
\label{2}
\end{equation}
其中$R_{n}=\frac{h^2}{2}u^{''}(\xi)$.

令
\begin{equation}
L[u_{n};h]=u_{n+1}-u_{n}-hf(t_{n},u_{n})
\end{equation}
取$u_{n}=u(t_{n})$则
\begin{equation}
R_{n}=L[u(t_{n});h]=u(t_{n+1})-u(t_{n})-hu^{'}(t_{n})
\end{equation}
则称$R_{n}$为局部截断误差.

我们称$e_{n}=u(t_{n})-u_{n}$为整体误差,将(\ref{2})和(\ref{1})相减,知$e_{n}$满足误差方程:
\begin{equation}
e_{n+1}=e_{n}+h[f(t_{n},u(t_{n}))-f(t_{n},u_{n})]+R_{n}
\label{3}
\end{equation}
因$f(t,u)$关于$u$满足Lipschitz条件(\ref{3})有:
\begin{align}
|e_{n+1}|&\le|e_{n}|+Lh|e_{n}|+R\\
&=(1+Lh)|e_{n}+R
\end{align}
其中$R=\max|R_{n}|$以此类推,得:
\begin{align}
|e_{n}|&\le (1+Lh)|e_{n-1}|+R\le (1+Lh)^2|e_{n-2}|+(1+Lh)R+R\\
&\le \cdots\le (1+Lh)^n|e_{0}|+R\sum_{j=0}^{n-1}(1+Lh)^j\\
&=(1+Lh)^n|e_{0}|+\frac{R}{Lh}[(1+Lh)^n-1]\\
&\le e^{nLh}|e_{0}|+\frac{R}{Lh}(e^{nLh}-1)
\end{align}
由于$t_{n}=t_{0}+nh\le T,n=(t_{n}-t_{0})/h$,于是
\begin{equation}
|e_{n}|\le e^{L(T-t_{0})}|e_{0}|+\frac{R}{Lh}(e^{L(T-t_{0})}-1),~~~n=1,\cdots,N
\end{equation}
右端依赖初始误差$e_{0}$和局部截断误差的界$R$.对欧拉法,可取$R=Ch^2$(C是与n无关的常数).

若取$e_{0}=0$(取$u_{0}=u(t_{0})$),则
\begin{equation}
|e_{n}|\le CL^{-1}e^{L(T-t_{0})}h
\end{equation}
所以$e_{n}=O(h)$,比局部截断误差低一阶.

类比的方法,将$u(t_{n})$在$u({t_{n+1}})$处泰勒展开可得:
\begin{align}
u(t_{n})&=u(t_{n+1})-hu^{'}(t_{n+1})+\frac{h^2}{2}u^{''}(t_{n+1})-\frac{h^3}{6}u^{'''}(\xi)\\&=u(t_{n+1})-hu^{'}(t_{n+1})+R_{n}
\end{align}
略去二阶小量,用$u_{n}$逼近$u(t_{n})$得到向后欧拉:
\begin{equation}
u_{n+1}=u_{n}+hf(t_{n+1},u_{n+1})
\end{equation}


\par
~\\
~\\
~\\
\textbf{冯诺依曼条件}

$G(\rho h,\tau )$表示增长因子,则差分格式的稳定的必要性是存在与$\tau$无关的常数M使
\begin{equation}
|G(\rho h,\tau )|\le 1+M\tau ~~~~~\rho=1,2,\cdots,J-1
\end{equation}

差分格式稳定$\Leftrightarrow G(ph,\tau)$一致有界$\Leftrightarrow$von Neumann条件成立.

例如:一维热传导方程
\begin{equation}
\frac{\partial u}{\partial t}=\alpha\frac{\partial^2 u}{\partial^2 x}
\end{equation}
在时间上进行向前差分,空间中心差分得:
\begin{equation}
u_{j}^{n+1}=u_{j}^{n}+\gamma(u_{j+1}^n-2u_{j}^n+u_{j-1}^n)
\end{equation}
其中网比$\gamma=\frac{\alpha\tau}{h^2}$
进行傅里叶变换:$u_{j}^n=\hat{u}^{n}e^{ikx}$得
\begin{equation}
\hat{u}^{n+1}e^{ikx}=(\gamma e^{ik(x+h)}+(1-2\gamma)e^{ikx}+\gamma e^{ik(x-h)})\hat{u}^{n}
\end{equation}
消去$e^{ikx}$得:
\begin{equation}
\hat{u}^{n+1}=G(\rho h,\tau)\hat{u}^n
\end{equation}
则增长因子:
\begin{align}
G(\rho h,\tau)&=(1-2\gamma)+\gamma (e^{ikh}+e^{-ikh})\\
&=1-2\gamma (1-coskh)\\
&=1-4rsin^{2}\frac{kh}{2}
\end{align}
为了使$G(\rho h,\tau)$满足 von Neumann 条件:
\begin{align}
-1-M\tau\le G(\rho h,\tau)\le 1+M\tau
\end{align}
即
\begin{align}
-1-M\tau\le 1-4\gamma sin^2\frac{kh}{2}\le 1+M\tau
\end{align}
解得$0\le \gamma\le \frac{1}{2}$.当且仅当网比$\gamma\le \frac{1}{2}$成立.

\par 
~\\
~\\
~\\
~\\

设初值问题
\begin{equation*}
\begin{cases}
f^{'}(u)=F(t,f(u)),\\
f(u(t_{0}))=f(u_{0})
\end{cases}
\end{equation*}
求解
\begin{equation}
\dot{u}=f(u)+g(u),
\label{12}
\end{equation}
f显式处理,g隐式处理

f为向前欧拉,则有:
\begin{equation}
f(u_{n})=f(u_{n-1})+hF(t_{n-1},f(u_{n-1}))
\end{equation}

g为向后欧拉,则有:
\begin{equation}
g(u_{n})=g(u_{n-1})+hF(t_{n},g(u_{n}))
\end{equation}

则
\begin{align}
\dot{u}_{n}&=f(u_{n})+g(u_{n})\\&=f(u_{n-1})+g(u_{n-1})+h(F(t_{n-1},f(u_{n-1}))+F(t_{n},g(u_{n}))\\&=\dot{u}_{n-1}+h(F(t_{n-1},f(u_{n-1}))+F(t_{n},g(u_{n}))
\end{align}
左右两边关于t积分,得$u_{n}=u_{n-1}+Ch(f(u_{n-1})+g(u_{n}))$

二级隐式龙格库塔可以写成:
\begin{equation}
\begin{cases}
g(u_{n})=g(u_{n-1})+h(b_{1}k_{1}+b_{2}k_{2})\\
k_{1}=F(t_{n-1},g(u_{n-1})+ha_{11}k_{1})\\
k_{2}=F(t_{n-1},g(u_{n-1})+ha_{21}k_{1}+ha_{22}k_{2})
\end{cases}
\end{equation}
三级显式龙格库塔可以写成:
\begin{equation}
\begin{cases}
f(u_{n})=f(u_{n-1})+h(\hat{b}_{1}\hat{k}_{1}+\hat{b}_{2}\hat{k}_{2}+\hat{b}_{3}\hat{k}_{3})\\
\hat{k}_{1}=F(t_{n-1},f(u_{n-1}))\\
\hat{k}_{2}=F(t_{n-1},f(u_{n-1})+h\hat{a}_{21}\hat{k}_{1})\\
\hat{k}_{3}=F(t_{n-1},f(u_{n-1})+h\hat{a}_{31}\hat{k}_{1}+h\hat{a}_{32}\hat{k}_{2})
\end{cases}
\end{equation}
则
\begin{align}
\dot{u}_{n}&=f(u_{n})+g(u_{n})\\
&=f(u_{n-1})+g(u_{n-1})+h(b_{1}k_{1}+b_{2}k_{2}+\hat{b}_{1}\hat{k}_{1}+\hat{b}_{2}\hat{k}_{2}+\hat{b}_{3}\hat{k}_{3})\\&
=\dot{u}_{n-1}+h(b_{1}k_{1}+b_{2}k_{2}+\hat{b}_{1}\hat{k}_{1}+\hat{b}_{2}\hat{k}_{2}+\hat{b}_{3}\hat{k}_{3})
\end{align}
对t积分,则有${u}_{n}=u_{n-1}+Ch(b_{1}k_{1}+b_{2}k_{2}+\hat{b}_{1}\hat{k}_{1}+\hat{b}_{2}\hat{k}_{2}+\hat{b}_{3}\hat{k}_{3})$

如果显式龙格库塔是三级三阶的,则需要满足的条件是
\begin{equation}
\begin{cases}
b_{1}+b_{2}+b_{3}=1\\
c_{2}b_{2}+c_{3}b_{3}=\frac{1}{2}\\
c_{2}^2b_{2}+c_{3}^2b_{3}=\frac{1}{3}\\
c_{2}a_{32}b_{3}=\frac{1}{6}
\end{cases}
\end{equation}
这是含两参数的三级三阶方法.现取$b_{3}=(2-\sqrt{2})/2,b_{1}=0.$则可算出
\begin{equation}
\begin{cases}
c_{2}=(2-\sqrt{2})/2\\
c_{3}=1\\
a_{32}=(3+2\sqrt{2})/3\\
a_{31}=-2\sqrt{2}/3\\
b_{2}=\sqrt{2}/2
\end{cases}
\end{equation}

如果隐式龙格库塔是二级二阶的,则需要满足的条件是:
\begin{equation}
\begin{cases}
b_{1}+b_{2}=1\\
b_{1}c_{1}+b_{2}c_{2}=\frac{1}{2}
\end{cases}
\end{equation}
现取$b_{2}=c_{1}=(2-\sqrt{2})/2$则可算出
\begin{equation}
\begin{cases}
b_{1}=\sqrt{2}/2\\
c_{2}=1
\end{cases}
\end{equation}
由于这是单隐对角隐式龙格库塔,则有$a_{11}=a_{22}=(2-\sqrt{2})/2,a_{21}=\sqrt{2}/2$.
~\\
~\\
~\\
求初值问题形如:
\begin{equation}
\sum_{j=0}^{k}\alpha_{j}y_{m+j}=h\sum_{j=0}^{k}\beta_{j}f_{m+j}
\label{10}
\end{equation}
其中$h>0$是积分步长,$k\ge 1$是方程的步数。

定义:若对某点$\bar{h}\in \bar{C}$($\bar{C}$表示扩张的复平面)$\bar{h}=\lambda h$.特征多项式$\prod(\xi ,\bar{h})$的每个根的模都严格地小于1,则称方法$(\ref{10})$关于点$\bar{h}$是绝对稳定的,在$\bar{C}$上,使方法(\ref{10})绝对稳定的一切点$\bar{h}$的集合,称为该方法的绝对稳定区域,或简称稳定域.

以$\xi_{1}(\bar{h}),\xi_{2}(\bar{h}),\cdots,\xi_{k}(\bar{h})$表示多项式$\prod(\xi,\bar{h})$的k个根,则方法的绝对稳定区域可表示为:
\begin{equation}
S=\{ \bar{h}\in\bar{C}:|\xi_{j}(\bar{h})|<1,~j=1,2,\cdots,k \}
\end{equation}

定义:数值积分公式称为A稳定的,如果这个积分公式的稳定区域包含复开左半平面.

(Dahquist提出)方法(\ref{10})称为是A-稳定的,如果满足:$S$为绝对稳定区域,
\begin{equation*}
S\supset C_{\_}=\{\bar{h}\in C|Re\bar{h}<0\}.
\end{equation*}

线性多步法说是A稳定的,如果将它用于模型问题:
\begin{equation}
u^{'}(t)=\lambda u(t)
\label{19}
\end{equation}
绝对稳定区域就是左负平面$\bar{C}$,其中$\lambda$是复数.
(\ref{10})相应的特征多项式为:
\begin{equation}
\prod(\xi,\bar{h})=\sum_{j=0}^{k}\alpha_{j}\lambda^{j}-\bar{h}\sum_{j=0}^{k}\beta_{j}\lambda^{j}=\rho(\lambda)-\bar{h}\sigma(\lambda)
\end{equation}
设$\lambda_{i}(i=1,2,\cdots,k)$是特征方程的根,则下列表述等价:

(1)线性多步法A稳定;

(2)$Re\bar{h}<0\Rightarrow|\lambda_{i}|<1,i=1,2,\cdots,k;$

(3)$|\lambda|\ge 1\Rightarrow Re\bar{h}(\lambda)\ge 0$

举例:向后欧拉:$u_{n+1}=u_{n}+hf_{n+1}$
由于$\rho(\lambda)=\lambda-1,\sigma(\lambda)=\lambda$则
\begin{align}
Re\bar{h}(\lambda)&=Re\frac{\lambda-1}{\lambda}\\&=\frac{|\lambda|^2-|\lambda|cos\theta}{|\lambda|^2}\\&=\frac{|\lambda|(|\lambda|-cos\theta)}{|\lambda|^2}
\end{align}
显然当$|\lambda|\ge 1$时,$Re\bar{h}\ge 0$,故向后欧拉A稳定.

方法(\ref{10})称为是L-稳定的,如果它是A-稳定的,且在$\infty$点极端稳定。

一个方法是L-稳定的,如果满足A-稳定和
\begin{equation*}
\lim\limits_{z\to\infty}R(z)=0
\end{equation*}

龙格库塔法求解标量线性模型方程(\ref{19})得到差分方程:
\begin{equation}
y_{n+1}=R(z)y_{n}
\end{equation}
s级单隐Runge-Kutta法有形如
\begin{equation}
R(z)=\frac{P(z)}{(1-\lambda z)^s}
\end{equation}
的稳定函数.$P(z)=\sum_{i=0}^{s}(\sum_{j=0}^{i}\frac{(-1)^j}{(i-j)!}\binom{s}{j}\lambda^j)z^i$.
得:
\begin{equation}
R(\infty)=\lim\limits_{z\to 0}R(z)=\frac{1-4\lambda+2\lambda^2}{2\lambda^2}
\end{equation}

由$R(\infty)=0$,得$\lambda=(2\pm\sqrt{2})/2$.

\end{CJK}
\end{document}