\subsubsection{均匀鞍点}
通常可以通过分析方法定位均匀的鞍点,例如,规范模型A场理论的方程(5.3)简化为
\begin{equation}
\frac{1}{u_0}w^*(r)+i\tilde{\rho}(\br;[iw^*])=0
\end{equation}
这里,在构造函数导数时应用了方程(4.75)中的密度算子$\tilde{\rho}(\br;[iw^*])$。在无界系统中,或在受周期性边界条件影响的立方体单元中,该方程的唯一解析解是均匀的,即$w^*(\br)=w^*$。从方程(3.25)和(3.22)得出$q(\br,s;[iw^*])=\exp(-iw^*s)$和$Q[iw^*]=\exp(-iw^*N)$以及方程(4.75)中$\tilde{\rho}(\br;[iw^*])=nN/V$,其与平均段密度$\rho_0$一致。因此,方程(5.11)简化为
\begin{equation}
w^*=-iu_0\rho_0
\end{equation}
由于排除了体积参数$u_0$和平均段密度$\rho_0$都是实数和正数的情况,因此势场的鞍点值$w^*$位于复数$w$平面的负虚轴上。这类似于第5.1.2节中讨论的玩具(toy)模型的情况。

在均匀鞍点处的规范模型A的有效哈密顿量由等式(4.72)估计为$H[w^*]=(1/2)u_0\rho_0^2V$。由此得出,聚合物溶液的Helmholtz自由能的平均场近似值由下面式子给出
\begin{equation}
A(n,V,T)=A_0+\frac{k_BT}{2}u_0\rho_0^2V
\end{equation}
这里$A_0=-k_BT\ln\calZ_0$是非相互作用聚合物的理想气体的自由能。在平均场近似中,自由能的过量部分仅由聚合物链段之间的平均相互作用能量产生,没有考虑它们的空间相关性。

对大规范模型A重复此分析是很有用的。鞍点方程相当于
\begin{equation}
\frac{\delta H_G[w]}{\delta w(\br)}\bigg|_{w=w^*}=\frac{1}{u_0}w^*(\br)+i\tilde{\rho}_G(\br;[iw^*])=0
\end{equation}
其中$\tilde{\rho}_G$是方程(4.76)中给出的分段密度算子。对于均匀$w^*(\br)=w^*$,可以发现
\begin{equation}
\tilde{\rho}_G(\br;[iw^*])=zN\exp(-iw^*N)=\frac{\left\langle n\right\rangle N}{V}\equiv\rho_0
\end{equation}
其中$\left\langle n\right\rangle=zVQ[iw^*]=zV\exp(-iw^*N)$是体积$V$中聚合物的平均数,而$\rho_0$是平均链段密度。由此可见,平均场满足以下超越方程
\begin{equation}
iw^*\exp(iw^*N)=u_0zN
\end{equation}
因为右端是实的和正的,所以该方程有唯一的解$w^*$,其位于任何聚合物活性$z$的负虚轴上。

与大规范模型A的热力学的关联通过以下公式将渗透压$\Pi$与$\calZ_G$联系起来
\begin{equation}
\begin{aligned}
\beta\Pi&=\frac{1}{V}\ln\calZ_G(z,V,T)\\
&\approx-\frac{1}{V}H_G[w^*]=\frac{\rho_0}{N}+\frac{1}{2}u_0\rho_0^2
\end{aligned}
\end{equation}
其中在上面公式的第二行中应用了平均场近似。因此,我们看到渗透压的平均场表达式包括与聚合物平均密度成比例的理想气体项的总和,$\rho_0/N$和与单位体积自由能中相应项的相互作用项组成。这些当然是众所周知的结果(de Gennes,1979)。对于第四章中介绍的其他模型的均匀平均场解,可以推导出类似的表达式。
\subsection{进一步近似}
更令人感兴趣的是鞍点方程(5.3)的非均匀解。通常需要数值分析的方法来得到这种平均场解的精确描述。数值方法的讨论推迟到第5.3节;在这里,我们考虑进一步的分析近似,使不均匀的鞍点方程易于处理。这些近似基于第3.4节中提出的方案,用于估计外部势场中单链的统计特性。
\subsubsection{弱不均匀性-RPA}
可以与平均场近似一起应用的一种重要的近似类型是弱不均匀性扩展。在方程(5.3)的解几乎是一致的情况下,我们可以用方程(3.113)类比得出
\begin{equation}
w^*(\br)=w_0+\omega^*(\br)
\end{equation}
其中$w_0\equiv(1/V)\int w^*(\br)\mathrm{d}\br$是体积平均的平均场势场,假设偏差$\omega^*(\br)$与$w_0$相比在任何地方都很小,可以遵循3.4.1节的流程来产生弱的不均匀性扩展。这种扩展在聚合物文献中通常称为随机相近似,或RPA(de Gennes,1969;de Gennes,1979)。我们用规范模型A来说明它。

规范模型A的鞍点方程在方程(5.11)中给出。代替方程(5.18)并应用方程(3.134)和(4.75)导出以下扩展:
\begin{equation}
u_0^{-1}\omega^*(\br)+\rho_0N\int g_D(\left|\br-\br'\right|)\omega^*(\br')\mathrm{d}\br'+O((\omega^*)^2)=0
\end{equation}
其中方程(5.12)用于取消主导齐次项,$g_D$是方程(3.133)的Debye函数。方程(5.19)的唯一解是$\omega^*(\br)=0$,这与我们之前的声明一致,即规范模型A的整体或具有周期性边界条件的单元的唯一鞍点是均匀解$w^*(\br)=w_0$。

RPA扩展还可用来观察密度函数$F[\rho]$的形式,这是第4.10节DFT形式的核心,在平均场近似中,规范模型A的方程(4.207)简化为
\begin{equation}
\rho(\br)\approx\tilde{\rho}(\br;[iw^*+J])
\end{equation}
我们用$\rho(\br)$的简写代替$\left\langle\hat{\rho}(\br)\right\rangle_J$。该等式确定由任意外部势场$J(\br)$产生的平均段密度$\rho(\br)$。此外,在平均场近似中,方程(4.205)的配分函数简化为$\calZ_C[J]\approx\calZ_0\exp(-H[w^*,J])$,在这两个表达式中,鞍点$w^*(\br)$由下式确定
\begin{equation}
\frac{\delta H[w,J]}{\delta w(\br)}\bigg|_{w=w^*}=u_0^{-1}w^*(\br)+i\tilde{\rho}(\br;[iw^*+J])=0
\end{equation}
通过选择$J$,使其幅值较弱并且具有较好的体积平均值,即$(1/V)\int J(\br)\mathrm{d}\br=0$,方程(5.21)的右端可以类似于方程(5.19)的RPA扩展得出。在$J$的领先秩序,
\begin{equation}
\begin{aligned}
\int[u_0^{-1}\delta(\br-\br')&+\rho_0Ng_D(\left|\br-\br'\right|)]\omega^*(\br')\mathrm{d}\br'\\
&=i\rho_0N\int g_D(\left|\br-\br'\right|)J(\br')+O(J^2)
\end{aligned}
\end{equation}
该结果的傅里叶变换产生一下关于$\omega^*$和$J$的公式:
\begin{equation}
\hat{\omega}^*(\bk)=\frac{iu_0\rho_0N\hat{g}_D(x)}{1+u_0\rho_0N\hat{g}_D(x)}\hat{J}(\bk)+O(J^2)
\end{equation}
其中$x=k^2R_g^2$是无量纲的平方波数,其具有未受扰动的回旋半径的平方,$R_g^2=Nb^2/6$。

下一步是使用方程(5.23)和(3.131)扩展方程(4.206)中给出的函数$H[w^*,J]$。这将得出:
\begin{equation}
H[w^*,J]=H_0-\frac{1}{2V}\sum\limits_{\bk}\frac{\rho_0N\hat{g}_D(x)}{1+u_0\rho_0N\hat{g}_D(x)}\hat{J}(\bk)\hat{J}(-\bk)+O(J^3)
\end{equation}
其中$H_0\equiv(1/2u_0)Vw_0^2+w_0Nn$是对哈密顿量的均匀贡献。构造自由能函数$F[\rho]$所需的最后一步是通过方程(4.199)利用勒让德变换将$J$变为$\rho$,即
\begin{equation}
F[\rho]=-\ln\calZ_0+H[w^*,J]-\int J(\br)\rho(\br)\mathrm{d}\br
\end{equation}
这个变换需要扩展方程(5.20)来建立$J$和$\rho$之间的关系。应用方程(3.134)得
\begin{equation}
\widehat{\Delta\rho}(\bk)=-\frac{\rho_0N\hat{g}_D(x)}{1+u_0\rho_0N\hat{g}_D(x)}\hat{J}(\bk)+O(J^2)
\end{equation}
其中$\Delta\rho(\br)\equiv\rho(\br)-\rho_0$是单体密度场的不均匀部分。结合方程(5.24)-(5.26)得到所需的自由能泛函
\begin{equation}
F[\rho]=F_0+\frac{1}{2V}\sum\limits_{\bk}\left(\frac{1}{\rho_0N\hat{g}_D(x)}+u_0\right)\widehat{\Delta\rho}(\bk)\widehat{\Delta\rho}(-\bk)+O(\Delta\rho^3)
\end{equation}
其中$F_0$是均匀流体的平均场自由能。

方程(5.27)是对应于模型A的弱非均匀聚合物溶液的自由能(以$k_BT$为单位)的表达式。在适用平均场近似且密度不均匀性小的程度上是有效的。正如将在第6章讨论的那样,平均场近似适用于模型A,其浓度足够高满足
\begin{equation}
C\gg B\equiv\frac{u_0N^2}{R_g^3}
\end{equation}
其中$C=nR_g^3/V$是在方程(5.5)中引入的无量纲链浓度,$B$是无量纲消除体积参数。

方程(5.27)的一个有用的应用是估计聚合物溶液的均相相对于小幅度密度扰动的稳定性。通过检查二次系数可以确定稳定性。
\begin{equation}
\hat{\Gamma}_2(k)=\frac{1}{\rho_0N\hat{g}_D(k^2R_g^2)}+u_0
\end{equation}
作为波数$k=\left|\bk\right|$的函数。由于$\hat{g}_D(x)$是$x$的单调递减函数,因此,$\hat{\Gamma}_2(k)$的最小值与$k=0$一致。均匀相或螺旋线的稳定极限因此对应于
\begin{equation}
\hat{\Gamma}_2(0)=\frac{1}{\rho_0N}+u_0=0
\end{equation}

















































































