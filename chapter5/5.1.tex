\subsection{平均场近似}
前一章演示了如何从简单和复杂流体的粒子模型构建统计场理论。如何在平衡状态下分析这些场理论,并提取有关聚合物流体结构和热力学性质的有用信息,是目前研究讨论的热点。\\

第四章的场论模型将相关的配分函数一般表示为一个或多个化学势场上的函数积分$w(\mathrm{r})$,即\\
\begin{gather}
	\mathcal{Z}=\int \mathcal{D}w \ \exp(-H[w])
\end{gather}
其中$H[w]$是一个有效的哈密顿量,它是场变量的非局部泛函,通常是复的(不是严格实的)。$H[w]$的形式依赖于特定的相互作用和用于构建场理论的链模型,因此对聚合物的结构、分子量、多分散性、组成等都很敏感。为了计算某些可观测$G$的集合平均,我们应用公式\\
\begin{gather}
	\langle G[w]\rangle = \mathcal{Z}^{-1}\int \mathcal{D}\omega G[w]\ \exp(-H[w])
\end{gather}
通过与$\mathcal{Z}$的热力学联系公式计算自由能和导数量,需要计算方程(5.1)给出的泛函积分。同样地,利用相应的密度算子$G[w]$计算方程(5.2),将流体结构计算为两个泛函积分的比值。\\

实际上,对于非平凡的三维流体模型,这些函数积分都不能用封闭的形式来计算。但是,有几种选择:\\

1.\ \ 生成精确场理论的数值逼近。\\

2.\ \ 采用解析近似来简化场理论,然后用分析方法计算感兴趣的量。\\

3.\ \ 采用解析近似来简化理论,然后用数值方法从简化理论中提取信息。\\
\\
第一种策略,我们称之为“场理论模拟”(FTS)技术,是第六章的主题。本章讨论了第二和第三种方法。\\
\subsubsection{平均场近似:一般想法}
最重要的解析近似技术是平均场近似,这在聚合物物理学文献中也被称为自洽场理论(Edwards,1965;de Gennes,1969)。这一技术在许多物理环境中得到了广泛的应用,也许最显著的是在相变理论中(AMIT,1984;Parisi,1988;Goldenfeld,1992)。在这种情况下,平均场近似等于假设单场构型(configuration)$w^*(\mathrm{r})$控制方程(5.1)和(5.2)中的泛函积分。这种场构型是通过要求$H[w]$相对$w(\mathrm{r})$的变化是平稳的得到的,即\\
\begin{gather}
	\left.\frac{\delta H[w]}{\delta w(\mathrm{r})}\right|_{w=w^*} =0
\end{gather}
在得到这个方程的“平均场”势$w^*(r)$后,可以完成近似通过\\
\begin{gather}
	\mathcal{Z} \approx \exp(-H[w^*]),\quad \langle G[w]\rangle \approx G[w^*]
\end{gather}
如果$\mathcal{Z}$代表正则配分函数,则Helmholtz自由能由$\beta A=−ln\mathcal{Z}=H[w^*]$立即得到。\\

在平均场近似下,除特殊构型$w^*(\mathcal{r})$外,势场的所有构型都被忽略。对于原子或小分子流体,这种忽略所有“场波动(filed fluctuations)”的近似通常是相当差的。这是因为在液体密度下,原子或小分子的典型配位数(typical coordination number)很低,$\sim 10$,所以当粒子的位置发生变化时,粒子的电位会发生很大的波动。事实上,这些强的局域场波动是产生描述液体原子尺度结构的密度关联(dendity correlations)的原因(Hansen和McDonald,1986)。\\

由于配位数仍然很小,在聚合物溶液或原子尺度的熔体中,平均场近似也是不准确的。然而,在\textbf{介观尺度}上,由于聚合物线圈相互渗透的能力,情况发生了质的变化。从介观角度看,聚合物流体中有效配位数的一个有用定义是\\
\begin{gather}
	C=\rho_c R_g^3
\end{gather}
其中$\rho_c=n/V$是链的平均数密度,$R_g$表示聚合物的旋转半径。因此,C对应于一条兴趣链所占体积(估计为$R_g^3$)的(其他)聚合物链的平均数。这一有效的配位数可以是非常大的浓缩溶液或聚合物熔体的高分子质量。例如,在均聚物熔体中,$ρ_c=1/(v_0N)$,其中$v_0$是每个统计段所占的体积,N是每条链的段数。由于聚合物在$R_g=b(N/6)^{1/2}$的熔体中是渐近理想的,因此$C\sim (b^3/v_0)N{1/2}$。因此,在熔体条件下,有效配位数随着分子量的平方根的增大而增大。因此,在介观尺度下,每种聚合物的环境波动都随着分子量的增加而减小,因为势场的变化是通过接触越来越多的周围链来平均的。这构成了一个标准论点,即平均场近似对于高分子量聚合物的浓缩溶液或熔体是准确的(deGennes,1979年)。\\


平均场近似的实现需要求解方程(5.3)来获得“平均场”$w^*(r)$。对于实际的模型,这类方程是非线性的,非局部的,并且不服从封闭形式的解析解.然而,对于包含精确场理论的复杂泛函积分,平均场方程仍然是一个很大的简化。在讨论求解平均场方程的数值和近似解析策略之前,先讨论与复平面场理论的解析结构有关的一些一般性问题,以及方程(5.3)多解的存在性和意义。\\
\subsubsection{场理论的解析结构}


